\documentclass[uplatex,dvipdfmx,landscape]{jsarticle}
\usepackage[dvipdfm,papersize={58cm,31cm},margin=2cm,noheadfoot]{geometry}
\usepackage{metalogo}
\newcommand*\TikZ{Ti\textit{k}Z}
\usepackage{tikz}
\usetikzlibrary{arrows.meta}
\usetikzlibrary{calc}
\usetikzlibrary{decorations.pathmorphing}
\usepackage{enumitem}
\pagestyle{empty}


\tikzset{input node/.style={draw,rectangle,ultra thick,fill=blue!30!white}}
\tikzset{relay node/.style={draw,rectangle,fill=red!20!white}}
\tikzset{output node/.style={draw,rectangle,double,outer sep=1pt,fill=green!30!white}}
\tikzset{speed priority/.style={->,>={Stealth[scale=1.2]},dashed,thick,color=blue}}
\tikzset{quality priority/.style={->,>={Stealth[scale=1.2]},thick,color=red,decorate,decoration={snake, pre length=5pt, post length=5pt, segment length=6pt, amplitude=1pt}}}
\tikzset{legend box/.style={minimum width=1cm,minimum height=\baselineskip}}

\def\fboxtitle#1{\par\noindent {\fboxrule1pt\fbox{\huge\sffamily\bfseries Ghostscript #1を利用している場合}}\nopagebreak\par}

\begin{document}
\baselineskip12pt
\parindent0pt

\fboxtitle{9.15 以上}

\vspace{5mm}

{\fboxsep8pt
\fbox{\parbox{42zw}{\baselineskip20pt\lineskiplimit12pt\lineskip4pt
{\Large\textgt{凡例}}\par
\makebox[3.5zw][c]{\tikz[baseline=-3pt]\node[legend box,input node] {};}\hspace{1zw}入力ファイル\par
\makebox[3.5zw][c]{\tikz[baseline=-3pt]\node[legend box,relay node] {};}\hspace{1zw}中間ファイル\par
\makebox[3.5zw][c]{\tikz[baseline=-3pt]\node[legend box,output node] {};}\hspace{1zw}出力ファイル\par
\makebox[3.5zw][c]{\tikz[baseline=-3pt]\draw[quality priority](0,0) -- (1,0);}\hspace{1zw}画質優先モードのみでの経路\par
\makebox[3.5zw][c]{\tikz[baseline=-3pt]\draw[speed priority](0,0) -- (1,0);}\hspace{1zw}速度優先モードのみでの経路\par
}}}

\vspace{-4cm}

\begin{center}
\begin{tikzpicture}[x=1.3cm,y=1.3cm,align=center,>={Stealth[scale=1.6]}]
\node[input node] (TeX) at (0,0) {\TeX};
\node[relay node] (DVI) at ($(TeX) + (0,-2)$) {DVI};
\draw[->] (TeX) -- node[left] {((u)p)\LaTeX} (DVI) ;

\node[input node] (PS) at ($(DVI) + (-2,0)$) {PS};
\draw[->] (DVI) -- node[above] {dvips} (PS) ;

\node[input node] (PDF) at ($(TeX) + (0,-4)$) {PDF};
\draw[->] (DVI) -- node[right] {dvipdfmx} (PDF) ;
\draw[->] (PS) -- node[left] {Ghostscript} (PDF) ;

\node[input node] (EPS) at ($(PDF) + (-4,0)$) {EPS};
\draw[->] (EPS) -- node[above] {Ghostscript} (PDF) ;
\coordinate (right above of PDF) at ($(PDF.north east) + (1,-0.1)$);
\coordinate (right below of PDF) at ($(PDF.south east) + (0, 0.1)$);
\draw[->] (TeX) -- (TeX -| right above of PDF) to [bend left=90] node[right] {pdf\LaTeX\\\XeLaTeX\\\LuaLaTeX} (right above of PDF) to (PDF.north east |- right above of PDF) ;


\node[relay node] (PDF with text without margin) at ($(PDF) + (-16,-3)$) {テキスト保持PDF\\(余白なし・透過)};
\draw[->] (PDF) -- node[left,xshift=-5pt,yshift=9pt] {Quartz APIで\\クロップ} (PDF with text without margin);

\node[relay node] (Outlined PDF without margin) at ($(PDF with text without margin) + (-5,0)$) {アウトライン化PDF\\(余白なし・透過)};
\draw[quality priority] (PDF with text without margin) -- node [above,xshift=4pt] {Ghostscript\textsuperscript{*2}} (Outlined PDF without margin);

\node[output node] (Bitmap) at ($(PDF with text without margin) + (0,-2)$) {ビットマップ画像};
\draw[speed priority] (PDF with text without margin) -- node[right,align=left] {Quartz APIでビットマップ化\\+余白付与(+背景塗り)} (Bitmap);
\draw[quality priority] (Outlined PDF without margin.south east) -- node [left,align=right,xshift=-10pt,yshift=-10pt] {Quartz APIでビットマップ化\\+余白付与(+背景塗り)} (Bitmap.north west);

\node[output node] (Multipage TIFF) at ($(Bitmap) + (-4,-2)$) {マルチページTIFF};
\draw[->] (Bitmap.south west) -- node[above,xshift=-10pt,yshift=2pt] {tiffutil} (Multipage TIFF.north east);

\node[output node] (Animation GIF) at ($(Bitmap) + (0,-2)$) {アニメーションGIF};
\draw[->] (Bitmap) -- node[right] {Quartz API} (Animation GIF);


\node[output node] (PDF with text) at ($(PDF) + (0,-3)$) {テキスト保持PDF\\(余白あり・透過)};
\draw[->] (PDF) -- node[right,align=left] {Quartz APIでクロップ\\+余白付与} (PDF with text);

\node[output node] (Multipage PDF with text) at ($(PDF with text) + (0,-4)$) {マルチページ\\テキスト保持PDF\\(透過)};
\draw[->] (PDF with text) -- node[right] {Quartz API} (Multipage PDF with text);

\node[output node] (SVG) at ($(PDF with text) + (4,-2)$) {テキスト保持SVG\\(透過)};
\draw[->] (PDF with text) -- node[right,xshift=-8pt,yshift=8pt] {mudraw} (SVG);



\node[output node] (Filled PDF with text) at ($(PDF with text) + (11,0)$) {テキスト保持PDF\\(余白あり・非透過)};
\draw[->] (PDF with text) --node[above] {Quartz APIで背景塗り} (Filled PDF with text);

\node[output node] (Filled EPS with margin) at ($(Filled PDF with text) + (5,0)$) {バイナリ形式EPS\\(余白あり・非透過)};
\draw[->] (Filled PDF with text) -- node[above] {Ghostscript\textsuperscript{*1}} (Filled EPS with margin);

\node[output node] (Filled Multipage PDF with text) at ($(Filled PDF with text) + (0,-4)$) {マルチページ\\テキスト保持PDF\\(非透過)};
\draw[->] (Filled PDF with text) -- node[right] {Quartz API} (Filled Multipage PDF with text);


\node[output node] (Filled SVG) at ($(Filled PDF with text) + (-4,-2)$) {テキスト保持SVG\\(非透過)};
\draw[->] (Filled PDF with text) -- node[left,xshift=8pt,yshift=8pt] {mudraw} (Filled SVG);

\node[relay node] (Single page PDF) at ($(PDF) + (6,0)$) {単一ページPDF\\(余白なし・透過)};
\draw[->] (PDF) -- node[below] {Quartz APIでクロップ} (Single page PDF);

\node[output node] (EPS without margin) at ($(Single page PDF) + (5,0)$) {バイナリ形式EPS\\(余白なし・透過)};
\draw[->] (Single page PDF) -- node[below] {Ghostscript\textsuperscript{*1}} (EPS without margin);

\node[output node] (EPS with margin) at ($(EPS without margin) + (5,0)$) {バイナリ形式EPS\\(余白あり・透過)};
\draw[->] (EPS without margin) -- node[below] {BB編集} (EPS with margin);

\node[output node] (Outlined PDF with margin) at ($(PDF with text) + (-7,0)$) {アウトライン化PDF};
\path[->] (Outlined PDF with margin) edge [loop above,out=135,in=45,distance=50] node[above] {Quartz APIで背景塗り} (Outlined PDF with margin);
\draw[->] (PDF with text) -- node[above,midway,xshift=4pt]{Ghostscript\textsuperscript{*2}} (Outlined PDF with margin); 

\node[output node] (EPS2) at ($(Outlined PDF with margin) + (4,-2)$) {テキスト形式EPS};
\draw[->] (Outlined PDF with margin) -- node[right,yshift=4pt] {pdftops} (EPS2);

\node[output node] (EMF) at ($(EPS2) + (0,-2)$) {EMF};
\draw[->] (EPS2) -- node[left,align=right] {eps2emf\textsuperscript{*3}\\(改造版pstoedit)} (EMF);

\node[output node] (Multipage Outlined PDF) at ($(Outlined PDF with margin) + (0,-4)$) {マルチページ\\アウトライン化PDF};
\draw[->] (Outlined PDF with margin) -- node[right] {Quartz API} (Multipage Outlined PDF);

\node[output node] (Outlined SVG) at ($(EPS2) + (-8,0)$) {アウトライン化SVG};
\draw[->] (Outlined PDF with margin) -- node[left,yshift=4pt] {mudraw} (Outlined SVG);

\node[output node] (Animation SVG) at ($(EPS2) + (-8,-2)$) {アニメーションSVG};
\draw[->] (Outlined SVG) -- node[right] {SVG編集} (Animation SVG);


\end{tikzpicture}
\end{center}

\vspace{5mm}


\textgt{\Large 補足情報}

{\baselineskip18pt
\begin{itemize}[leftmargin=2zw]
\item\relax[*1]のGhostscirpt実行においては,eps2writeデバイスが用いられる。
\item\relax[*2]のGhostscirpt実行においては,pdfwriteデバイスが用いられる。
\item\relax[*1]および[*2]のGhostscirpt実行における \texttt{-r} オプションの値は,画質優先モードでは20016固定,速度優先モードでは解像度レベル設定に従う。
\item\relax[*1]のGhostscript実行においては,出力されるEPSのBoundingBox値が誤っている場合がある。そこで,eps2writeで生成されたEPSに対しては,Ghostscriptのbboxデバイスで取得されるBoundingBox値によって常に上書きするようにしている。
\item\relax[*3]のeps2emf実行前に,パスのアウトライン化を行うように生成EPSを加工しておく。
\item 「元のページサイズを維持」がONの場合,[*1]のGhostscript実行においては,出力されるEPSのBB値を,変換前のPDFの指定されたPageBoxの値で上書きする。
\item 「元のページサイズを維持」がONの場合,「Quartz APIでクロップ」の過程では,変換前のPDFの指定されたPageBox(余白付与の場合はそれに余白を加えたもの)をMediaBoxにしたPDF(左下が原点,他のBoxは非明示)を生成する。
\item\relax[*1]で用いられるGhostscriptのeps2writeデバイスは,\TikZ のshadowsライブラリを用いた場合など,特定の種類の図は苦手としている。[*1]においてeps2writeデバイスでEPSに変換すると,ぼやけたビットマップ画像にされてしまう。
\item 余白は原則としてbp単位が用いられるが,設定でpx単位を選んでいて,かつビットマップ画像出力の場合は,Quartz APIによるビットマップ化実行時にpx単位の余白が付与される。
\end{itemize}
}

\newpage %%%%%%%%%%%%%%%%%%%%%%%%%%%%%%%%%%%%%%%%%%%%%%%%%%%%%%%%%%%%%%%%%%%%%%%%%%%%%%%%%%%%%%%%%%%%%%%%%%%%%%%%%%%%%%%%%%%%%%%%%%%%%%

\fboxtitle{9.15 未満}


\vspace{5mm}

{\fboxsep8pt
\fbox{\parbox{42zw}{\baselineskip20pt\lineskiplimit12pt\lineskip4pt
{\Large\textgt{凡例}}\par
\makebox[3.5zw][c]{\tikz[baseline=-3pt]\node[legend box,input node] {};}\hspace{1zw}入力ファイル\par
\makebox[3.5zw][c]{\tikz[baseline=-3pt]\node[legend box,relay node] {};}\hspace{1zw}中間ファイル\par
\makebox[3.5zw][c]{\tikz[baseline=-3pt]\node[legend box,output node] {};}\hspace{1zw}出力ファイル\par
\makebox[3.5zw][c]{\tikz[baseline=-3pt]\draw[speed priority](0,0) -- (1,0);}\hspace{1zw}速度優先モードのみでの経路\par
}}}

\vspace{-4cm}

\begin{center}
\begin{tikzpicture}[x=1.3cm,y=1.3cm,align=center,>={Stealth[scale=1.6]}]
\node[input node] (TeX) at (0,0) {\TeX};
\node[relay node] (DVI) at ($(TeX) + (0,-2)$) {DVI};
\draw[->] (TeX) -- node[left] {((u)p)\LaTeX} (DVI) ;

\node[input node] (PS) at ($(DVI) + (-2,0)$) {PS};
\draw[->] (DVI) -- node[above] {dvips} (PS) ;

\node[input node] (PDF) at ($(TeX) + (0,-4)$) {PDF};
\draw[->] (DVI) -- node[right] {dvipdfmx} (PDF) ;
\draw[->] (PS) -- node[left] {Ghostscript} (PDF) ;

\node[input node] (EPS) at ($(PDF) + (-4,0)$) {EPS};
\draw[->] (EPS) -- node[above] {Ghostscript} (PDF) ;
\coordinate (right above of PDF) at ($(PDF.north east) + (1,-0.1)$);
\draw[->] (TeX) -- (TeX -| right above of PDF) to [bend left=90] node[right] {pdf\LaTeX\\\XeLaTeX\\\LuaLaTeX} (right above of PDF) to (PDF.north east |- right above of PDF) ;


\node[relay node] (PDF with text without margin) at ($(PDF) + (-14,-3)$) {テキスト保持PDF\\(余白なし・透過)};
\draw[speed priority] (PDF) -- node[left,xshift=-5pt,yshift=9pt] {Quartz APIで\\クロップ} (PDF with text without margin);

\node[output node] (Bitmap) at ($(PDF with text without margin) + (0,-2)$) {ビットマップ画像\\(速度優先)};
\draw[speed priority] (PDF with text without margin) -- node[left,align=right] {Quartz APIでビットマップ化\\+余白付与(+背景塗り)} (Bitmap);

\node[output node] (Multipage TIFF) at ($(Bitmap) + (-4,-2)$) {マルチページTIFF\\(速度優先)};
\draw[speed priority] (Bitmap.south west) -- node[above,xshift=-10pt,yshift=2pt] {tiffutil} (Multipage TIFF.north east);

\node[output node] (Animation GIF) at ($(Bitmap) + (0,-2)$) {アニメーションGIF\\(速度優先)};
\draw[speed priority] (Bitmap) -- node[right] {Quartz API} (Animation GIF);


\node[output node] (PDF with text) at ($(PDF) + (-10,-3)$) {テキスト保持PDF\\(余白あり・透過)};
\draw[->] (PDF) -- node[right,align=left,xshift=15pt,yshift=-3pt] {Quartz APIでクロップ\\+余白付与} (PDF with text.north east);

\node[output node] (Multipage PDF with text) at ($(PDF with text) + (0,-4)$) {マルチページ\\テキスト保持PDF\\(透過)};
\draw[->] (PDF with text) -- node[left] {Quartz API} (Multipage PDF with text);

\node[output node] (SVG) at ($(PDF with text) + (2,-2)$) {テキスト保持\\SVG(透過)};
\draw[->] (PDF with text) -- node[right,xshift=-5pt,yshift=8pt] {mudraw} (SVG);

\node[output node] (Filled PDF with text) at ($(PDF with text) + (7,0)$) {テキスト保持PDF\\(余白あり・非透過)};
\draw[->] (PDF with text) --node[above] {Quartz APIで背景塗り} (Filled PDF with text);

\node[output node] (Filled Multipage PDF with text) at ($(Filled PDF with text) + (0,-4)$) {マルチページ\\テキスト保持PDF\\(非透過)};
\draw[->] (Filled PDF with text) -- node[right,near end] {Quartz API} (Filled Multipage PDF with text);


\node[output node] (Filled SVG) at ($(Filled PDF with text) + (-2,-2)$) {テキスト保持\\SVG(非透過)};
\draw[->] (Filled PDF with text) -- node[left,xshift=5pt,yshift=8pt] {mudraw} (Filled SVG);

\node[relay node] (EPS for EMF) at ($(PDF) + (0,-2)$) {テキスト形式EPS};
\draw[->] (PDF) -- node[right] {pdftops} (EPS for EMF);

\node[relay node] (PDF for EMF) at ($(EPS for EMF) + (3.5,0)$) {PDF};
\draw[->] (EPS for EMF) -- node[above] {epstopdf\textsuperscript{*3}} (PDF for EMF);

\node[relay node] (PDF2 for EMF) at ($(PDF for EMF) + (0,-1.7)$) {テキスト保持PDF};
\draw[->] (PDF for EMF) -- node[left,align=right] {Quartz APIでクロップ\\+余白付与} (PDF2 for EMF);
\path[->] ($(PDF2 for EMF.north east) + (-6pt,0pt)$) edge [loop right,out=90,in=0,distance=35] node[right,xshift=-25pt,yshift=14pt] {Quartz APIで背景塗り} ($(PDF2 for EMF.north east) + (0pt,-6pt)$);



\node[relay node] (EPS2 for EMF) at ($(PDF2 for EMF) + (0,-1.7)$) {EPS};
\draw[->] (PDF2 for EMF) -- node[left] {Ghostscript\textsuperscript{*1}} (EPS2 for EMF);

\node[output node] (EMF) at ($(EPS2 for EMF) + (0,-1.7)$) {EMF};
\draw[->] (EPS2 for EMF) -- node[left,align=right] {eps2emf\\(改造版pstoedit)} (EMF);

\node[output node] (EPS without margin) at ($(Filled PDF with text) + (12,0)$) {EPS\\(余白なし・透過)};

\node[relay node] (PDF with text 2) at ($(EPS without margin) + (0,3)$) {テキスト保持PDF\\(余白なし・透過・単一ページ)};
\draw[->] (PDF) -- node[below] {Quartz APIでクロップ} (PDF with text 2);
\draw[->] (PDF with text 2) -- node[left] {Ghostscript\textsuperscript{*1}} (EPS without margin);

\node[output node] (Outlined PDF without margin) at ($(EPS without margin) + (6,0)$) {アウトライン化PDF};
\draw[->] (EPS without margin) -- node[above] {epstopdf\textsuperscript{*2}} (Outlined PDF without margin);
\path[->] ($(Outlined PDF without margin.north east) + (-6pt,0pt)$) edge [loop right,out=90,in=0,distance=35] node[right,xshift=2pt,yshift=2pt] {Quartz APIで余白付与} ($(Outlined PDF without margin.north east) + (0pt,-6pt)$);
\path[->] ($(Outlined PDF without margin.south east) + (0pt,6pt)$) edge [loop right,out=0,in=-90,distance=35] node[right,xshift=2pt,yshift=-2pt] {Quartz APIで背景塗り} ($(Outlined PDF without margin.south east) + (-6pt,0pt)$);

\node[output node] (Multipage Outlined PDF) at ($(Outlined PDF without margin) + (-2,-2.5)$) {マルチページ\\アウトライン化PDF};
\draw[->] (Outlined PDF without margin) -- node[left] {Quartz API} (Multipage Outlined PDF);

\node[output node] (Outlined SVG) at ($(Outlined PDF without margin) + (2,-2.5)$) {アウトライン化SVG};
\draw[->] (Outlined PDF without margin) -- node[right,yshift=3pt] {mudraw} (Outlined SVG);

\node[output node] (Animated SVG) at ($(Outlined SVG) + (0,-1.5)$) {アニメーションSVG};
\draw[->] (Outlined SVG) -- node[right] {SVG編集} (Animated SVG);


\node[output node] (Bitmap2) at ($(Outlined PDF without margin) + (0,3)$) {ビットマップ画像\\(画質優先)};
\draw[->] (Outlined PDF without margin) -- node[right,align=left] {Quartz APIでビットマップ化\\+余白付与(+背景塗り)} (Bitmap2);

\node[output node] (Multipage TIFF 2) at ($(Bitmap2) + (-2,2)$) {マルチページTIFF\\(画質優先)};
\draw[->] (Bitmap2.north west) -- node[left,yshift=-5pt] {tiffutil} (Multipage TIFF 2);

\node[output node] (Animation GIF 2) at ($(Bitmap2) + (2,2)$) {アニメーションGIF\\(画質優先)};
\draw[->] (Bitmap2.north east) -- node[right,yshift=-5pt] {Quartz API} (Animation GIF 2);

\node[output node] (EPS with margin) at ($(EPS without margin) + (0,-2.5)$) {EPS\\(余白あり・透過)};
\draw[->] (EPS without margin) -- node[left] {BB編集} (EPS with margin);

\node[output node] (Filled EPS with margin) at ($(Filled PDF with text) + (3,-2)$) {EPS\\(余白あり・非透過)};
\draw[->] (Filled PDF with text) -- node[right,yshift=5pt] {Ghostscript\textsuperscript{*1}} (Filled EPS with margin);

\end{tikzpicture}
\end{center}

\vspace{5mm}


\textgt{\Large 補足情報}

{\baselineskip18pt
\begin{itemize}[leftmargin=2zw]
\item\relax[*1]のGhostscirpt実行においては,epswriteデバイスが用いられる。
\item\relax[*1]のGhostscirpt実行における \texttt{-r} オプションの値は,画質優先モードでは20016固定,速度優先モードでは解像度レベル設定に従う。
\item\relax[*1]のGhostscript実行においては,出力されるEPSのBoundingBox値が誤っている場合がある。そこで,epswriteで生成されたEPSに対しては,Ghostscriptのbboxデバイスで取得されるBoundingBox値によって常に上書きするようにしている。
\item\relax[*2]のepstopdf実行においては,OS X 10.11 El Capitan 以上では \texttt{--hires} オプションありで,それ未満ではなしで実行する。これは El Capitan で修正された Quartz API の不具合に対応するための措置である。OS X 10.10 Yosemite 以下のOSでは,\texttt{--hires} オプションありで生成されたPDFを Quartz API にかけると端が欠けるという現象が発生する。
\item\relax[*3]のepstopdf実行前に,パスのアウトライン化を行うように生成EPSを加工しておく。
\item 「元のページサイズを維持」がONの場合,[*1]のGhostscript実行においては,出力されるEPSのBB値を,変換前のPDFの指定されたPageBoxの値で上書きする。
\item 「元のページサイズを維持」がONの場合,「Quartz APIでクロップ」の過程では,変換前のPDFの指定されたPageBox(余白付与の場合はそれに余白を加えたもの)をMediaBoxにしたPDF(左下が原点,他のBoxは非明示)を生成する。
\item\relax[*1]で用いられるGhostscriptのepswriteデバイスは,\TikZ のshadowsライブラリを用いた場合など,特定の種類の図は苦手としている。[*1]においてepswriteデバイスでEPSに変換すると,無数のパスに分割されてエラーを起こす原因となる。そのような図は,Ghostscriptによって綺麗にアウトライン化することはできないので,Ghostscriptを経由しない経路(テキスト保持PDF,テキスト保持SVG,速度優先モードでのビットマップ画像)で代用できないか検討してほしい。
\item 余白は原則としてbp単位が用いられるが,設定でpx単位を選んでいて,かつビットマップ画像出力の場合は,Quartz APIによるビットマップ化実行時にpx単位の余白が付与される。
\end{itemize}
}

\end{document}