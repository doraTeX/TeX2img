\documentclass[uplatex,dvipdfmx,a3paper,landscape]{jsarticle}
\usepackage[dvipdfm,margin=2cm,noheadfoot]{geometry}
\usepackage{metalogo}
\usepackage{tikz}
\usetikzlibrary{arrows.meta}
\usetikzlibrary{calc}
\usepackage{enumitem}
\pagestyle{empty}
\begin{document}
\baselineskip12pt
\parindent0pt

\tikzset{input node/.style={draw,rectangle,ultra thick}}
\tikzset{relay node/.style={draw,rectangle}}
\tikzset{output node/.style={draw,rectangle,double,outer sep=1pt}}
\tikzset{speed priority/.style={->,>={Stealth[scale=1.2]},dashed,thick}}
\tikzset{legend box/.style={minimum width=1cm,minimum height=\baselineskip}}

\fbox{\parbox{20zw}{\baselineskip18pt\lineskiplimit12pt\lineskip4pt
{\Large\textgt{凡例}}\par
\tikz[baseline=-3pt]\node[legend box,input node] {};\hspace{1zw}入力ファイル\par
\tikz[baseline=-3pt]\node[legend box,output node] {};\hspace{1zw}出力ファイル\par
\tikz[baseline=-3pt]\draw[speed priority](0,0) -- (1,0);\hspace{1zw}速度優先モードのみでの経路\par
}}

\begin{center}
\begin{tikzpicture}[x=1.3cm,y=1.3cm,align=center,>={Stealth[scale=1.6]}]
\node[input node] (TeX) at (0,0) {\TeX};
\node[draw,rectangle] (DVI) at ($(TeX) + (0,-2)$) {DVI};
\draw[->] (TeX) --  node[left] {((u)p)\LaTeX} (DVI) ;
\node[input node] (PS) at ($(DVI) + (-2,0)$) {PS};
\draw[->] (DVI) --  node[above] {dvips} (PS) ;
\node[input node] (PDF) at ($(TeX) + (0,-4)$) {PDF};
\draw[->] (DVI) --  node[right] {dvipdfmx} (PDF) ;
\draw[->] (PS) --  node[left] {Ghostscript} (PDF) ;
\node[input node] (EPS) at ($(PDF) + (-4,0)$) {EPS};
\draw[->] (EPS) --  node[above] {Ghostscript} (PDF) ;
\draw[->] (TeX) -- ++ (1,0) to [bend left=90]  node[right] {pdf\LaTeX\\\XeLaTeX\\\LuaLaTeX} ($(PDF) + (1,0)$) to (PDF) ;
\node[output node] (PDF with text) at ($(PDF) + (-8,-2)$) {テキスト保持PDF\\(余白あり)};
\draw[->] (PDF) -- node[left,align=left,xshift=-10pt] {pdf\TeX でクロップ\\+余白付与} (PDF with text.north east);
\node[output node] (SVG) at ($(PDF with text) + (0,-2)$) {SVG};
\draw[->] (PDF with text) -- node[left] {mudraw} (SVG);
\node[relay node] (EPS2) at ($(PDF with text) + (-4,0)$) {EPS};
\draw[->] (PDF with text) -- node[above] {Ghostscript\textsuperscript{*1}} (EPS2);
\node[output node] (EMF) at ($(EPS2) + (0,-2)$) {EMF};
\draw[->] (EPS2) -- node[left] {eps2emf\\(改造版pstoedit)} (EMF);
\node[relay node] (PDF with text without margin) at ($(PDF with text) + (6,0)$) {テキスト保持PDF\\(余白なし)};
\draw[speed priority] (PDF) -- node[left,xshift=-5pt,yshift=-2pt] {pdf\TeX で\\クロップ} (PDF with text without margin);
\node[output node] (Bitmap) at ($(PDF with text without margin) + (0,-2)$) {ビットマップ画像\\(速度優先)};
\draw[speed priority] (PDF with text without margin) -- node[left,align=right] {Quartz API でビットマップ化\\+余白付与} (Bitmap);
\node[output node] (EPS without margin) at ($(PDF with text without margin) + (5,0)$) {EPS\\(余白なし)};
\draw[->] (PDF) -- node[right,yshift=6pt] {Ghostscript\textsuperscript{*1}} (EPS without margin.north west);
\node[output node] (Outlined PDF without margin) at ($(EPS without margin) + (4,0)$) {アウトライン化PDF\\(余白なし)};
\draw[->] (EPS without margin) -- node[above] {epstopdf} (Outlined PDF without margin);
\node[output node] (Bitmap2) at ($(Outlined PDF without margin) + (7,0)$) {ビットマップ画像\\(画質優先)};
\draw[->] (Outlined PDF without margin) -- node[above] {Quartz API でビットマップ化\\+余白付与} (Bitmap2);
\node[output node] (Outlined PDF with margin) at ($(Outlined PDF without margin) + (0,-2)$) {アウトライン化PDF\\(余白あり)};
\draw[->] (Outlined PDF without margin) -- node[left] {pdf\TeX} (Outlined PDF with margin);
\node[output node] (EPS with margin) at ($(EPS without margin) + (0,-2)$) {EPS\\(余白あり)};
\draw[->] (EPS without margin) -- node[left] {BB編集} (EPS with margin);
\node[relay node] (PDF2) at ($(EPS with margin) + (0,-2)$) {アウトライン化PDF\\(余白あり)};
\draw[->] (EPS with margin) -- node[left] {epstopdf\textsuperscript{*2}} (PDF2);
\node[output node] (Text EPS) at ($(PDF2) + (0,-2)$) {テキスト形式EPS};
\draw[->] (PDF2) -- node[left] {pdftops\textsuperscript{*2}} (Text EPS);
\end{tikzpicture}
\end{center}

\vspace{5mm}


\textgt{\Large 補足事項}

{\baselineskip18pt
\begin{itemize}[leftmargin=2zw]
\item\relax[*1] の Ghostscirpt 実行においては,Ghostscript 9.15 以上では eps2write デバイスが,それ未満では epswrite が用いられる。
\item\relax[*1] の Ghostscirpt 実行における \texttt{-r} オプションの値は,画質優先モードでは 20016 固定,速度優先モードでは解像度レベル設定に従う。
\item\relax[*1] の Ghostscript 実行においては,epswrite / eps2write どちらのデバイスであっても,出力される EPS の BoundingBox 値が誤っている場合がある。そこで,eps(2)write で生成されたEPSに対しては,Ghostscript の bbox デバイスで取得される BoundingBox 値によって常に上書きするようにしている。
\item\relax[*2] の経路は Ghostscirpt 9.15 以上の環境で「テキスト形式EPS出力」を選んだときのみ実行される。Ghostscirpt 9.15 未満の場合は epswrite デバイスで得られたEPSをそのまま最終出力とする。
\item 余白は原則としてbp単位が用いられるが,設定でpx単位を選んでいて,かつビットマップ画像出力の場合は,Quartz API によるビットマップ化実行時にpx単位の余白が付与される。
\end{itemize}
}

\end{document}