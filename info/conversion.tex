\documentclass[uplatex,dvipdfmx,landscape]{jsarticle}
\usepackage[dvipdfm,papersize={60cm,31cm},margin=2cm,noheadfoot]{geometry}
\usepackage{metalogo}
\newcommand*\TikZ{Ti\textit{k}Z}
\usepackage{tikz}
\usetikzlibrary{arrows.meta}
\usetikzlibrary{calc}
\usetikzlibrary{decorations.pathmorphing}
\usepackage{enumitem}
\pagestyle{empty}
\begin{document}
\baselineskip12pt
\parindent0pt

\tikzset{input node/.style={draw,rectangle,ultra thick,fill=blue!30!white}}
\tikzset{relay node/.style={draw,rectangle,fill=red!20!white}}
\tikzset{output node/.style={draw,rectangle,double,outer sep=1pt,fill=green!30!white}}
\tikzset{speed priority/.style={->,>={Stealth[scale=1.2]},dashed,thick,color=blue}}
\tikzset{fill background/.style={->,>={Stealth[scale=1.2]},thick,color=red,decorate,decoration={snake, pre length=5pt, post length=5pt, segment length=6pt, amplitude=1pt}}}
\tikzset{legend box/.style={minimum width=1cm,minimum height=\baselineskip}}

{\fboxsep8pt
\fbox{\parbox{42zw}{\baselineskip20pt\lineskiplimit12pt\lineskip4pt
{\Large\textgt{凡例}}\par
\makebox[3.5zw][c]{\tikz[baseline=-3pt]\node[legend box,input node] {};}\hspace{1zw}入力ファイル\par
\makebox[3.5zw][c]{\tikz[baseline=-3pt]\node[legend box,relay node] {};}\hspace{1zw}中間ファイル\par
\makebox[3.5zw][c]{\tikz[baseline=-3pt]\node[legend box,output node] {};}\hspace{1zw}出力ファイル\par
\makebox[3.5zw][c]{\tikz[baseline=-3pt]\draw[speed priority](0,0) -- (1,0);}\hspace{1zw}速度優先モードのみでの経路\par
\makebox[3.5zw][c]{\tikz[baseline=-3pt]\draw[fill background](0,0) -- (1,0);}\hspace{1zw}非透過アウトライン化PDFまたは非透過アニメーションSVG生成時のみの経路\par
}}}

\vspace{-4cm}

\begin{center}
\begin{tikzpicture}[x=1.3cm,y=1.3cm,align=center,>={Stealth[scale=1.6]}]
\node[input node] (TeX) at (0,0) {\TeX};
\node[relay node] (DVI) at ($(TeX) + (0,-2)$) {DVI};
\draw[->] (TeX) --  node[left] {((u)p)\LaTeX} (DVI) ;

\node[input node] (PS) at ($(DVI) + (-2,0)$) {PS};
\draw[->] (DVI) --  node[above] {dvips} (PS) ;

\node[input node] (PDF) at ($(TeX) + (0,-4)$) {PDF};
\draw[->] (DVI) --  node[right] {dvipdfmx} (PDF) ;
\draw[->] (PS) --  node[left] {Ghostscript} (PDF) ;

\node[input node] (EPS) at ($(PDF) + (-4,0)$) {EPS};
\draw[->] (EPS) --  node[above] {Ghostscript} (PDF) ;
\coordinate (right of PDF) at ($(PDF.north east) + (1,-0.1)$);
\draw[->] (TeX) --  (TeX -| right of PDF) to [bend left=90]  node[right] {pdf\LaTeX\\\XeLaTeX\\\LuaLaTeX} (right of PDF) to (PDF.north east |- right of PDF) ;


\node[relay node] (PDF with text without margin) at ($(PDF) + (-12,-3)$) {テキスト保持PDF\\(余白なし)};
\draw[speed priority] (PDF) -- node[left,xshift=-5pt,yshift=9pt] {pdf\TeX で\\クロップ} (PDF with text without margin);

\node[output node] (Bitmap) at ($(PDF with text without margin) + (0,-2)$) {ビットマップ画像\\(速度優先)};
\draw[speed priority] (PDF with text without margin) -- node[left,align=right] {Quartz API でビットマップ化\\+余白付与} (Bitmap);

\node[output node] (Multipage TIFF) at ($(Bitmap) + (-4,-2)$) {マルチページTIFF\\(速度優先)};
\draw[speed priority] (Bitmap.south west) -- node[above,xshift=-10pt] {tiffutil} (Multipage TIFF.north east);

\node[output node] (Animation GIF) at ($(Bitmap) + (0,-2)$) {アニメーションGIF\\(速度優先)};
\draw[speed priority] (Bitmap) -- node[right] {Quartz API} (Animation GIF);


\node[output node] (PDF with text) at ($(PDF) + (-7,-3)$) {テキスト保持PDF\\(余白あり・透過)};
\draw[->] (PDF) -- node[right,align=left,xshift=15pt,yshift=-3pt] {pdf\TeX でクロップ\\+余白付与} (PDF with text.north east);

\node[output node] (Multipage PDF with text) at ($(PDF with text) + (0,-4.5)$) {マルチページ\\テキスト保持PDF\\(透過)};
\draw[->] (PDF with text) -- node[left] {Quartz API} (Multipage PDF with text);

\node[output node] (SVG) at ($(PDF with text) + (4,-1)$) {SVG\\(透過)};
\draw[->] (PDF with text) -- node[right,xshift=-8pt,yshift=6pt] {mudraw} (SVG);

\node[relay node] (EPS2) at ($(PDF with text) + (4,-2.5)$) {EPS\\(透過)};
\draw[->] (PDF with text) -- node[below,xshift=-18pt,yshift=-4pt] {Ghostscript\textsuperscript{*1}} (EPS2);

\node[output node] (EMF) at ($(EPS2) + (0,-2)$) {EMF\\(透過)};
\draw[->] (EPS2) -- node[left,align=right] {eps2emf\\(改造版pstoedit)} (EMF);

\node[output node] (Filled PDF with text) at ($(PDF with text) + (10,0)$) {テキスト保持PDF\\(余白あり・非透過)};
\draw[->] (PDF with text) --node[above] {Quartz API で背景塗り} (Filled PDF with text);

\node[output node] (Filled Multipage PDF with text) at ($(Filled PDF with text) + (0,-4.5)$) {マルチページ\\テキスト保持PDF\\(非透過)};
\draw[->] (Filled PDF with text) -- node[right] {Quartz API} (Filled Multipage PDF with text);


\node[output node] (Filled SVG) at ($(Filled PDF with text) + (-4,-1)$) {SVG\\(非透過)};
\draw[->] (Filled PDF with text) -- node[left,xshift=8pt,yshift=6pt] {mudraw} (Filled SVG);

\node[relay node] (Filled EPS2) at ($(Filled PDF with text) + (-4,-2.5)$) {EPS\\(非透過)};
\draw[->] (Filled PDF with text) -- node[below,xshift=18pt,yshift=-4pt] {Ghostscript\textsuperscript{*1}} (Filled EPS2);

\node[output node] (Filled EMF) at ($(Filled EPS2) + (0,-2)$) {EMF\\(非透過)};
\draw[->] (Filled EPS2) -- node[right,align=left] {eps2emf\\(改造版pstoedit)} (Filled EMF);




\node[output node] (EPS without margin) at ($(Filled PDF with text) + (5,0)$) {EPS\\(余白なし・透過)};
\draw[->] (PDF) -- node[right,yshift=6pt] {Ghostscript\textsuperscript{*1}} (EPS without margin.north west);

\node[relay node] (PDF with text 2) at ($(EPS without margin) + (0,3)$) {テキスト保持PDF\\(余白なし・透過・単一ページ)};
\draw[fill background] (PDF) -- node[above] {pdf\TeX} (PDF with text 2);
\draw[fill background] (PDF with text 2) -- node[right] {Ghostscript\textsuperscript{*1}} (EPS without margin);

\node[output node] (Outlined PDF without margin) at ($(EPS without margin) + (6,0)$) {アウトライン化PDF\\(余白なし・透過)};
\draw[->] (EPS without margin) -- node[above] {epstopdf\textsuperscript{*3}} (Outlined PDF without margin);

\node[output node] (Bitmap2) at ($(Outlined PDF without margin) + (0,3)$) {ビットマップ画像\\(画質優先)};
\draw[->] (Outlined PDF without margin) -- node[right,align=left] {Quartz API でビットマップ化\\+余白付与} (Bitmap2);

\node[output node] (Multipage TIFF 2) at ($(Bitmap2) + (-2,2)$) {マルチページTIFF\\(画質優先)};
\draw[->] (Bitmap2.north west) -- node[left,yshift=-5pt] {tiffutil} (Multipage TIFF 2);

\node[output node] (Animation GIF 2) at ($(Bitmap2) + (2,2)$) {アニメーションGIF\\(画質優先)};
\draw[->] (Bitmap2.north east) -- node[right,yshift=-5pt] {Quartz API} (Animation GIF 2);

\node[output node] (Outlined PDF with margin) at ($(Outlined PDF without margin) + (0,-3)$) {アウトライン化PDF\\(余白あり・透過)};
\draw[->] (Outlined PDF without margin) -- node[left] {pdf\TeX} (Outlined PDF with margin);


\node[output node] (Filled Outlined PDF without margin) at ($(Outlined PDF without margin) + (6,0)$) {アウトライン化PDF\\(余白なし・非透過)};
\draw[->] (Outlined PDF without margin) -- node[above] {Quartz APIで\\背景塗り} (Filled Outlined PDF without margin);

\node[output node] (Filled Outlined PDF with margin) at ($(Filled Outlined PDF without margin) + (0,-3)$) {アウトライン化PDF\\(余白あり・非透過)};
\draw[->] (Outlined PDF with margin) --node[above] {Quartz APIで\\背景塗り} (Filled Outlined PDF with margin);


\node[output node] (Multipage Outlined PDF) at ($(Outlined PDF with margin) + (0,-3)$) {マルチページ\\アウトライン化PDF\\(透過)};
\draw[->] (Outlined PDF with margin) -- node[left] {Quartz API} (Multipage Outlined PDF);
\draw[->] (Outlined PDF without margin.south west) to [bend right=45] node[left] {Quartz API} (Multipage Outlined PDF.north west);

\node[output node] (Filled Multipage Outlined PDF) at ($(Filled Outlined PDF with margin) + (0,-3)$) {マルチページ\\アウトライン化PDF\\(非透過)};
\draw[->] (Filled Outlined PDF with margin) -- node[left] {Quartz API} (Filled Multipage Outlined PDF);
\draw[->] (Filled Outlined PDF without margin.south east) to [bend left=45] node[right] {Quartz API} (Filled Multipage Outlined PDF.north east);

\node[output node] (Animation SVG) at ($(Filled Outlined PDF with margin)!.5!(Outlined PDF without margin)$) {アニメーションSVG};
\draw[->] (Filled Outlined PDF with margin) --node[right,yshift=5pt] {mudraw} (Animation SVG.south east);
\draw[->] (Outlined PDF with margin) --node[left,xshift=3pt,yshift=5pt] {mudraw} (Animation SVG.south west);
\draw[->] (Filled Outlined PDF without margin) --node[right,yshift=-5pt] {mudraw} (Animation SVG.north east);
\draw[->] (Outlined PDF without margin) --node[left,xshift=3pt,yshift=-5pt] {mudraw} (Animation SVG.north west);

\node[output node] (EPS with margin) at ($(EPS without margin) + (0,-2)$) {EPS\\(余白あり)};
\draw[->] (EPS without margin) -- node[left] {BB編集} (EPS with margin);
\draw[->] (Filled PDF with text) -- node[left,yshift=-5pt] {Ghostscript\textsuperscript{*1}} (EPS with margin);

\node[relay node] (PDF2) at ($(EPS with margin) + (0,-2)$) {アウトライン化PDF\\(余白あり)};
\draw[->] (EPS with margin) -- node[left] {epstopdf\textsuperscript{*2}} (PDF2);

\node[output node] (Text EPS) at ($(PDF2) + (0,-2)$) {テキスト形式EPS};
\draw[->] (PDF2) -- node[left] {pdftops\textsuperscript{*2}} (Text EPS);

\end{tikzpicture}
\end{center}

\vspace{5mm}


\textgt{\Large 補足情報}

{\baselineskip18pt
\begin{itemize}[leftmargin=2zw]
\item\relax[*1]のGhostscirpt実行においては,Ghostscript 9.15以上ではeps2writeデバイスが,それ未満ではepswriteが用いられる。
\item\relax[*1]のGhostscirpt実行における \texttt{-r} オプションの値は,画質優先モードでは20016固定,速度優先モードでは解像度レベル設定に従う。
\item\relax[*1]のGhostscript実行においては,epswrite / eps2writeどちらのデバイスであっても,出力されるEPSのBoundingBox値が誤っている場合がある。そこで,eps(2)writeで生成されたEPSに対しては,Ghostscriptのbboxデバイスで取得されるBoundingBox値によって常に上書きするようにしている。
\item 「元のページサイズを維持」がONの場合,[*1]のGhostscript実行においては,出力されるEPSのBB値を,変換前のPDFの指定されたPageBoxの値で上書きする。
\item 「元のページサイズを維持」がONの場合,「pdf\TeX でクロップ」の過程では,変換前のPDFの指定されたPageBox(余白付与の場合はそれに余白を加えたもの)をMediaBoxにしたPDF(左下が原点,他のBoxは非明示)を生成する。
\item\relax[*2]の経路はGhostscirpt 9.15以上の環境で「テキスト形式EPS出力」を選んだときのみ実行される。Ghostscirpt 9.15未満の場合はepswriteデバイスで得られたEPSをそのまま最終出力とする。
\item Ghostscriptのeps(2)writeデバイスは,\TikZ のshadowsライブラリを用いた場合など,特定の種類の図は苦手としている。Ghostscirpt 9.15未満のepswriteデバイスの場合は,無数のパスに分割されてエラーを起こす原因となる。Ghostscirpt 9.15以上のeps2writeデバイスの場合は,ビットマップ化されてしまう。そのような図は,Ghostscriptによって綺麗にアウトライン化することはできないので,Ghostscriptを経由しない経路(テキスト保持PDF,SVG,速度優先モードでのビットマップ画像)で代用できないか検討してほしい。
\item\relax[*3]のepstopdf実行においては,OS X 10.11 El Capitan 以上では \texttt{--hires} オプションありで,それ未満ではなしで実行する。これは El Capitan で修正された Quartz API の不具合に対応するための措置である。OS X 10.10 Yosemite 以下のOSでは,\texttt{--hires} オプションありで生成されたPDFを Quartz API にかけると端が欠けるという現象が発生する。
\item 余白は原則としてbp単位が用いられるが,設定でpx単位を選んでいて,かつビットマップ画像出力の場合は,Quartz APIによるビットマップ化実行時にpx単位の余白が付与される。
\end{itemize}
}

\end{document}